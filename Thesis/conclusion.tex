\subsection{مقدمه}

در این مستند، گزارش مختصری درباره مساله تولید خودکار شرح بر تصاویر و روش‌های پیشنهادی برای حل چالش‌های موجود در این مسیر را ارائه دادیم. مساله تولید خودکار شرح بر تصاویر، به معنای تولید جملات زبان طبیعی برای هر تصویر است به طوری‌که این جملات شامل سه شرط زیر باشند:
\begin{enumerate}
\item صحنه، اجسام موجود در تصویر، رابطه مکانی اجسام و اطلاعاتی از این دست که به درک تصویر کمک می‌کنند باید در جملات تولید شده وجود داشته باشند و دقیق و کامل باشند.
\item جملات تولید شده، خود، باید به لحاظ معنایی، دستور زبان و املایی صحیح بوده و نقصی نداشته باشند.\
\item جملات تولید شده باید با تصاویر مرتبط با خود، سازگاری داشته باشند.
\end{enumerate}

ایده‌های اولیه در این مسیر از پژوهش‌های موجود در زمینه ترجمه ماشین ایجاد شده است که در آن‌ها، ابتدا یک جمله ورودی از یک زبان مبدا، با استفاده از روش‌های مختلفی به یک بردار ویژگی تبدیل می‌شود و سپس در مرحله دوم، بردار ویژگی حاصل، با استفاده از روش‌های خاص دیگری به جملات زبان طبیعی به زبان مقصد، تبدیل می‌شوند. حال اگر به جای جمله از زبان مبدا، یک تصویر به این سامانه وارد شود و با روشی بتوان این تصویر را به همان بردار ویژگی نگاشت کرد، جمله نهایی معادل معنایی تصویر ورودی خواهد شد. با استفاده از این فرایند می‌توان به طور خودکار برای تصاویر، شرح مناسبی به زبان طبیعی تولید کرد.
\\
در این مسیر، دو چالش عمده در پیش رو وجود دارد که باید مرتفع شوند:

\begin{enumerate}
\item چالش درک صحنه
\\
فرایند استخراج اطلاعات بصری نهفته در تصویر و بازنمایی مناسب این اطلاعات را به گونه‌ای که بتواند برای پردازش‌های بعدی مناسب باشد، فرایند درک صحنه می‌نامند. روش‌های مختلف و متعددی برای حل این چالش، تاکنون مطرح شده‌اند. در این مساله‌، باید بتوان تصاویر ورودی را به نحوی موثر و مفید به فضای ویژگی‌ها نگاشت کرد به طوری‌که بازنمایی حاصل، بتواند در مرحله تولید جمله، منجر به تولید جملات معنادار و مناسب شود.
\item چالش تولید جمله
\\
تولید جملات به زبان طبیعی که علاوه بر صحت معنایی، دستور زبانی و املایی، قادر به توصیف و تفسیر اطلاعات غیر قابل تفسیر برای کاربران انسانی هستند، از جمله مهم‌ترین و پویاترین حوزه‌های پژوهشی در زمینه هوش مصنوعی است و توجه پژوهش‌گران بسیاری را به‌ خود جلب کرده است. در این مساله، باید بتوان بردار ویژگی حاصل از تصویر را که در مرحله درک صحنه تولید شده است، به نحوی کارا و موثر به یک جمله در زبان طبیعی نگاشت کرد.
\end{enumerate}


\subsection{درک صحنه}

اولین مرحله از فرایند تولید خودکار شرح برای تصاویر، مرحله درک صحنه است. در این مرحله، تصاویر ورودی تحت عملیات مختلفی به فضای معنایی نگاشت می‌شوند. فضای معنایی در این‌جا، می‌تواند فضای شامل میدان‌های اطلاعاتی از پیش تعیین شده (مانند فضای سه‌تایی‌های «جسم، رخداد، صحنه») یا فضای بردار ویژگی‌ها باشد.
\\
روش‌های مختلفی برای نگاشت تصویر ورودی به فضای معنایی ارائه شده است که به طور کلی می‌توان عموم آن‌ها را به دو بخش تقسیم کرد:
\begin{enumerate}
\item
روش‌های مبتنی بر مدل‌های گرافی احتمالاتی\\
در این روش‌ها با استفاده از مدل‌های استاندارد گرافی احتمالاتی موجود یا با ارائه یک مدل گرافی احتمالاتی، تصویر ورودی به فضای معنایی نگاشت می‌شود. در روش‌های مبتنی بر این مدل‌ها، با ارائه یک توزیع احتمال برای نقاط مختلف در فضای معنایی، محتمل‌ترین نقطه برای تصویر به عنوان نقطه نظیر تصویر، انتخاب می‌شود.
	\begin{enumerate}
		\item مدل میدان تصادفی مارکف\\

یک نمونه از روش‌های مبتنی بر مدل میدان تصادفی مارکف که برای درک صحنه از آن استفاده شده است، در پژوهش\cite{Farhadi2010every} 
ارائه شده است. درک صحنه در این پژوهش با ارائه یک سه‌تایی «جسم، فعالیت، صحنه» به‌ازای هر تصویر، تعریف شده است. مبتنی بر همین تعریف، یک مدل میدان تصادفی مارکف شامل سه گره که دو‌به‌دو به هم متصل هستند، تعریف شده است. هر یک از گره‌های موجود در این مدل، نماینده یکی از میدان‌های سه‌گانه تعریف شده در فضای معنایی هستند. با تعریف توابع پتانسیل مختلف روی هر گره و توابع پتانسیل مختلف روی هر یال، یک تابع توزیع توام برای تمام متغیرهای تصادفی موجود در مدل ارائه شده است.
\\
با محاسبه مقادیر پتانسیل برای تصاویر مختلف موجود در مجموعه‌آموزشی و با استفاده از یک ماشین بردار پشتیبان، بردارهای ویژگی شاخص برای هر گره محاسبه می‌شوند. از این بردارهای ویژگی بعدا برای انطباق تصاویر با مقادیر مختلف در هر گره استفاده می‌شود.
\\
در این پژوهش،‌ با یافتن نزدیک‌ترین همسایه‌های یک تصویر بر حسب معیار شباهت با بردارهای ویژگی شاخص  و میانگین‌گیری روی مقادیر هر گره، بهترین انطباق تصویر و نقاط فضای معنایی به‌دست می‌آید. به این ترتیب، برای هر تصویر ورودی، می‌توان نقطه نظیر در فضای معنایی را مشخص کرد.
		
		\item مدل میدان تصادفی شرطی\\
		در پژوهش\cite{fidler2013sentence} 
		یک مدل میدان تصادفی شرطی سلسله‌مراتبی برای درک صحنه ارائه شده است که شامل دو سطح انتزاع است. برای گره‌های موجود در هریک از سطوح انتزاع مدل، یک دسته متغیر تصادفی تعریف شده و برای کل مدل سه نوع تابع پتانسیل مختلف معرفی شده است.
		\\
		اولین دسته از توابع پتانسیل معرفی شده در این بخش، توابع پتانسیل قطعه‌بندی یگانی هستند که به منظور یکپارچه‌سازی نقاط داخل یک قطعه تعریف شده‌اند. توابع پتانسیل دیگری برای انطباق بین متغیرهای تصادفی موجود در بین دو سطح انتزاع تعریف شده‌اند که در صورت مغایرت مقادیر اختصاص داده‌شده به متغیرهای موجود بین دو سطح، مقدار $-\lambda$ و در غیر این صورت مقدار صفر دارند. این توابع در شرایطی که مقادیر متغیرهای موجود در دو سطح با هم یکسان نباشد، یک مقدار جریمه به تابع هدف اضافه می‌کنند. آخرین دسته از توابع پتانسیل مورد استفاده، برای انطباق تصویر با دسته تشخیص داده‌شده اجسام تعریف شده است که توسط فلزنسوالب ارائه شده و به روش دی پی ام مشهور است.
		
		\item سایر مدل‌های گرافی احتمالی
در پژوهش\cite{li2007and}، یک مدل گرافی احتمالی مولد برای نگاشت تصویر به فضای معنایی ارائه شده است. در این مدل، از دو سطح تصویر استفاده شده است؛ تصویر سطح جسم و تصویر سطح صحنه. برای تصویر سطح صحنه، یک متغیر تصادفی، بیان‌کننده دسته صحنه و برای تصویر سطح جسم دو متغیر تصادفی، بیان‌کننده دسته و شکل جسم، ارائه شده است. روابط بین متغیرهای تصادفی در این پژوهش، براساس نحوه تولید متغیرهای تصادفی و روابط منطقی موجود بین آن‌ها طراحی شده‌اند.
\\
تصویر ورودی در این پژوهش، ابتدا به نواحی کوچک 10*10 تقسیم می‌شود و مطابق با روش توضیح داده شده، مقدار توابع پتانسیل مختلف برای هر کدام از متغیرهای تصادفی، در هر ناحیه، محاسبه می‌شود. در این پژوهش، یک تابع احتمال شرطی برای متغیرهای تصادفی ارائه شده است که در مرحله استنتاج، با استفاده از روش تخمین بیشترین احتمال، برچسب‌های هر تصویر مشخص می‌شوند.
	\end{enumerate}

\item
روش‌های مبتنی بر استفاده از شبکه‌های عصبی کانولوشنی عمیق\\
در این روش‌ها، با ارائه یک شبکه عصبی کانولوشنی عمیق و تعریف کردن تابع هدف برای شبکه، تابع نگاشت تصویر و فضای معنا تشکیل می‌شود. پس از ارائه توابع هدف برای هر شبکه، با بهینه‌سازی آن تابع، پارامترهای موجود در شبکه آموزش داده می‌شوند.
\\
در پژوهش\cite{Girshick_2014_CVPR}، روشی ارائه شده است که طی آن یک تصویر، به نواحی کوچک‌تر تقسیم می‌شود به طوری‌که هر ناحیه به‌وجودآمده، به طور یکپارچه، حاوی یک جسم باشد و هر جسم تنها در یک ناحیه قرار بگیرد. این روش موسوم به روش \lr{RCNN} است. در این روش، دو ویژگی برای یک ناحیه‌بندی خوب در تصاویر ارائه شده است و پیرو این ویژگی‌ها، روشی برای طرح نواحی پیشنهادی در یک تصویر که دارای این دو ویژگی باشد، ارائه شده است.
\\
ویژگی مطرح شده اول برای ناحیه‌بندی تصاویر این است که، ناحیه‌های ایجاد شده در هر تصویر، می‌توانند در ابعاد مختلف وجود داشته باشند زیرا اجسام موجود در تصاویر، ممکن است اندازه و تعداد متفاوتی داشته باشند. دومین ویژگی برای یک ناحیه‌بندی خوب، این است که معیار انتخاب نواحی نباید برای تمام تصاویر، یکسان در نظر گرفته شود؛ زیرا معیارهای مختلف برای ناحیه‌بندی تصاویر در شرایط مختلف، رفتارهای متفاوتی از خود نشان می‌دهند. بنابراین باید از معیارهای مختلف برای تعیین نواحی استفاده نمود.
\\
در این پژوهش، ابتدا تصاویر مطابق با یک معیار اولیه، به مجموعه‌ای از نواحی اولیه تقسیم می‌شوند. سپس با استفاده از معیارهای مختلف مانند فضاهای رنگی مختلف،‌ معیارهای شباهت مختلف و نقاط اولیه متفاوت، با پیروی از یک روش حریصانه، نواحی کوچکتر که به یک‌دیگر شبیه‌تر هستند با هم ترکیب شده و نواحی بزرگتر را می‌سازند. نواحی ایجاد شده در این روش، سپس به یک شبکه عصبی کانولوشنی عمیق داده می‌شوند و برای هر ناحیه، یک بردار ویژگی 4096 بعدی ایجاد می‌شود که هر ناحیه با آن بازنمایی شود.
\\
در پژوهش \cite{karpathy2014deep} با استفاده از روش \lr{RCNN} و تعریف دو تابع هدف دیگر برای شبکه، روشی ارائه شده است که طی آن بتوان تصاویر و جملات را به طور دوطرفه به یک‌دیگر نگاشت کرد. توابع هدف تعریف شده در این پژوهش، دو تابع مختلف هستند. اولین تابع هدف، یک تابع هدف سراسری است. این تابع به این منظور تعریف شده است که تصاویر و جملاتی که مطابق با محاسبات شبکه عصبی ارائه شده، بیشترین شباهت را با یک‌دیگر دارند، در واقعیت هم شبیه‌ترین تصاویر و جملات به یک‌دیگر باشند. تابع هدف دوم برای این شبکه به این شکل تعریف شده است که نواحی استخراج شده از تصویر و عبارات استخراج شده از جملات که در روش ارائه شده، بیشترین شباهت را به یک‌دیگر دارند،‌ در واقعیت هم بیشترین شباهت و ارتباط را با یک‌دیگر داشته باشند.
\\
در این پژوهش، تصاویر ورودی با استفاده از روش \lr{RCNN} به نواحی مختلف تقسیم شده و ۱۹ ناحیه با بیشترین اطمینان از بین این نواحی انتخاب می‌شود. این ۱۹ ناحیه به همراه خود تصویر به عنوان ۲۰ تصویر مختلف مورد استفاده قرار می‌گیرند. جملات ورودی با استفاده از روشی که در فصل تولید جملات زبان طبیعی توضیح داده خواهد شد، به عبارات مختلف تقسیم می‌شوند و بین هر عبارت استخراج شده و هر یک از ۲۰ تصویر موجود، یک معیار شباهت محاسبه شده و بیشترین شباهت‌ها با هم درنظر گرفته می‌شوند. معیار شباهت مورد استفاده در این روش، ضرب داخلی بین بردارهای ویژگی عبارات و نواحی است. عبارات و نواحی که بیشترین شباهت را با یک‌دیگر دارند برای تولید جمله به مرحله بعد، ارسال می‌شوند.
\end{enumerate}

\subsection{تولید جمله}
چالش تولید جمله یکی از قدیمی‌ترین و پویاترین حوزه‌های فعالیتی و پژوهشی در هوش مصنوعی است که از اواسط قدن بیستم، توجه پژوهش‌گران بسیاری را به خود جلب کرده است. روش‌های مختلفی برای حل این مساله ارائه شده‌اند. از جمله این روش‌ها می‌توان به موارد زیر اشاره کرد:

\begin{enumerate}
\item تولید زبان طبیعی
\\
در این دسته از روش‌ها که از اواخر دهه بیستم تا کنون مورد استفاده قرار می‌گیرند، با طی فرایند در یک چارچوب کلی، سعی در تولید جملات مناسب دارند. این دسته از روش‌ها عموما برای تفسیر خودکار داده‌هایی که برای کاربران انسانی غیر قابل تفسیر هستند یا تفسیر دشواری دارند، به‌کار می‌روند. در این روش‌ها ابتدا با استفاده از ویژگی‌های مختلفی که در داده‌های ماشینی (داده‌های قابل تفسیر برای ماشین) کلمات مناسب انتخاب شده و سپس با استفاده از کلمات منتخب، عبارات زبانی (با جایگشت دادن کلمات و حذف عبارات غیر محتمل) تولید می‌شوند. سپس با اعمال قواعد دستور زبان و چینش عبارات زبانی در کنارهم، جملات نهایی تولید می‌شوند.
\item نزدیک‌ترین همسایه
\\
در این دسته از روش‌ها سعی می‌شود با ورود یک تصویر و نگاشت آن به فضای ویژگی‌ها، جمله‌ای از میان تمام جملات موجود در مجموعه‌داده انتخاب شود که بیشترین مشابهت با بردار ویژگی تصویر را دارد. بزرگ‌ترین مشکل در این روش‌ها انتخاب معیار مناسب برای محاسبه فاصله بین یک جمله و بردار ویژگی حاصل از تصویر است. در این روش، علاوه بر این‌که نیاز به وجود مجموعه‌داده وسیع و پوشا وجود دارد، ممکن است جمله نهایی، در انتها گویا و بیان‌کننده تمام جوانب تصویر نباشد و یا حتی با تصویر ورودی سازگاری نداشته باشد.
\\
برای حل این مشکل، سعی شد به جای استخراج نزدیک‌ترین جمله به تصویر موجود، مشابه‌ترین عبارات زبانی را با شکستن جملات موجود به عبارات سازنده، انتخاب کرده و با بهره‌گیری از روش تولید زبان طبیعی و یا روش‌های دیگر، چینش مناسبی از این عبارات را که در قالب یک یا چند جمله بیان شوند، تولید و به عنوان شرح بر تصویر، نمایش داد.
\item استفاده از قالب‌های زبانی آماده
\\
با وجود فعالیت‌های گوناگون در این زمینه و استفاده از روش‌های مختلف، هم‌چنان تضمین صحت جمله خروجی، کار دشواری است. به همین دلیل، سعی شد با ارائه یک یا چند قالب زبانی آماده و از پیش تعیین شده برای جملات، مانند قالب‌های جملات خبری، صحت جملات نهایی را تضمین کرد. در این دسته از روش‌ها، ویژگی‌های مختلفی از تصویر استخراج می‌شود که هریک از این ویژگی‌ها یا همه آن‌ها در کنار هم قادر هستند نقش‌هایی مانند «فعل»، «فاعل»، «مفعول» و موارد مشابه را در جمله متناظر با تصویر مشخص کنند. با استخراج کلمات مناسب و شناخت نقش آن‌ها در جمله و جای‌گذاری هر یک از این کلمات در مکان مناسب نقشی خود در قالب از پیش تعیین شده، جمله متناظر با هر تصویر استخراج می‌شود.
\item استفاده از شبکه‌های عصبی بازگشتی
\\
اگر چه استفاده از قالب‌های آماده و از پیش تعیین شده، تا حدی مشکلات موجود را حل می‌کند اما هم‌چنان چالش بزرگ‌تری حل نشده باقی مانده است. تولید جملات جدید، استفاده از کلمات و عبارات جدید و ابتکاری به طوری‌که علاوه بر تضمین رعایت دستور زبان، بتوان معنای جمله را نیز متضمن شد، چالش بزرگی است که در این مسیر کماکان وجود دارد.
\\
استفاده از شبکه‌های عصبی بازگشتی یکی از بهترین راه‌کارهای موجود برای حل این مشکل و رویارویی با این چالش هستند. استفاده از این شبکه‌ها در اواخر قرن بیستم در بین پژوهش‌گران رواج پیدا کرد تا جایی که ناپایداری الگوریتم پس‌انتشار خطا در آموزش این شبکه، راه را برای پژوهش‌های بعدی بست. پس از ارائه یک روش مناسب برای بهینه‌سازی بدون هسین در سال 2010، روشی برای آموزش یک شبکه عصبی بازگشتی موسوم به شبکه عصبی بازگشتی ضربی بر مبنای بهینه‌سازهای بدون هسین ارائه شد و نتایج آن به طور چشم‌گیری از روش‌های موجود بیشتر بود.
\\
ارائه شبکه عصبی بازگشتی ضربی، نقطه عطفی در مسیر علم در راستای حل چالش تولید جمله به حساب می‌آید. از حدود سال 2011 به بعد، استفاده از شبکه‌های عصبی بازگشتی برای تولید جمله به پویاترین و پرفعالیت‌ترین حوزه در مسائل مربوط به تولید جمله، به حساب می‌آید.
\end{enumerate}

خانم لی و همکارانش در سال 2015، در پژوهش \cite{karpathy2015deep}، با استفاده از شبکه‌های عصبی کانولوشنی عمیق و دو نوع از شبکه‌های عصبی بازگشتی موسوم به شبکه‌های عصبی بازگشتی مالتی‌مودال و شبکه‌های عصبی بازگشتی دوطرفه، روش مناسبی برای تولید خودکار شرح بر تصاویر ارائه داده است.
\\
در این پژوهش، ابتدا با بهره‌گیری از روش شبکه عصبی کانولوشنی ناحیه‌ای، نواحی از تصویر که شامل تصویر اجسام است، استخراج شده و با استفاده از یک شبکه عصبی کریشفسکی، بردار ویژگی برای هر ناحیه محاسبه می‌شود. سپس با بهره‌گیری از یک شبکه عصبی بازگشتی دوطرفه، عبارات مختلف از جمله استخراج و بردارهای ویژگی برای هر عبارت محاسبه می‌شود. سپس با استفاده از یک تابع هدف و مدل میدان تصادفی مارکف، هم‌ترازسازی بین نواحی و عبارات زبانی صورت گرفته و مدل آموزش داده می‌شود.
\\
 در ادامه با تخمین بهینه پارامترهای موجود و با استفاده از شبکه عصبی بازگشتی مالتی‌مودال، توزیع احتمال بهترین کلمه بعدی در یک جمله با داشتن کلمات قبلی و محتوای حاصل از بردار ویژگی محاسبه شده روی نواحی تصویر، محاسبه شده و بهترین کلمه بعدی تولید می‌شود. این کار تا جایی ادامه می‌یابد که شبکه، نشانه مخصوص پایان جمله را تولید کند.


\subsection{یادگیری عمیق}

از اواخر سال 2013، روش‌های مبتنی یادگیری عمیق، نظر بسیاری از پژوهش‌گرانی را که در حوزه تولید شرح متناظر تصویر فعالیت می‌کردند، به خود جلب نمودند. این دسته‌ از روش‌ها، به دلیل عمل‌کرد بهتری که از خود نشان دادند، توانستند جایگزین روش‌های گرافی احتمالاتی شوند. 
\\
از جمله پژوهش‌هایی که با استفاده از شبکه‌های عصبی عمیق اقدام به تولید شرح متناظر تصویر کردند، می‌توان به پژوهش خانم لی و همکارانش \cite{karpathy2015deep} در سال 2015  اشاره کرد. در مرحله آموزش این پژوهش، ابتدا با استفاده از روش شبکه عصبی کانولوشنی ناحیه‌ای که در بخش قبل، ارائه شد، نواحی تصویر که شامل تصویر یک جسم هستند، انتخاب شده و بردار ویژگی مربوط به هر کدام از این بخش‌ها، استخراج می‌شود.
\\
پس از این مرحله، بردار ویژگی مربوط به جملات موجود در مجموعه‌داده، توسط یک شبکه عصبی بازگشتی دوطرفه، استخراج می‌شود. برای این‌ کار، ابتدا بردار ویژگی مربوط به هر کلمه با استفاده از یک شبکه کلمه به بردار\lr{Word To Vec}، استخراج شده و به عنوان ورودی به شبکه بازگشتی دوطرفه داده می‌شوند. استفاده از شبکه بازگشتی دوطرفه این امکان را می‌دهد که تاثیر کلمات قبل و بعد از هر کلمه، در تولید بردار ویژگی جملات لحاظ شود.
\\
با بهینه‌سازی یک تابع انرژی روی این قسمت، شبکه عصبی بازگشتی دوطرفه و شبکه عصبی کانولوشنی با هم آموزش داده می‌شوند. از این طریق، بخش‌هایی از مدل که مربوط به تولید بردار ویژگی از جملات و استخراج نواحی تصاویر و بردار ویژگی مربوط به آن‌ها است، به طور کامل آموزش می‌بینند.
\\
در ادامه فرایند آموزش شبکه، با ارائه بردار ویژگی تولید شده توسط شبکه عصبی کانولوشنی آموزش دیده در بخش قبلی به یک شبکه عصبی بازگشتی دیگر، و ارائه جملات موجود در مجموعه‌داده به آن، شبکه عصبی بازگشتی را برای تولید جمله نهایی آموزش می‌دهیم.
\\
آزمایشات انجام شده روی این پژوهش، معیار \lr{BLEU} حاصل توسط روش را روی مجموعه‌داده \lr{MS COCO} در مقایسه با روش‌های دیگر ارزیابی کرده‌اند. در این آزمایشات، بهترین عمل‌کرد روش ارائه شده روی این مجموعه‌داده به امتیاز \lr{BLEU} برابر با 57.3 رسیده است و این در حالیست که روش \cite{mao2014explain} روی همان مجموعه‌داده به مقدار 55.0 رسیده است.
\\
یکی دیگر از روش‌های ارائه شده در این بخش، روشی است که در پژوهش \cite{chen2015mind} در سال 2015 ارائه شده است. در این روش، یک شبکه عصبی بازگشتی دوطرفه برای نگاشت جملات و تصاویر به یک‌دیگر استفاده شده است. مدل ارائه شده، قادر است با گرفتن تصویر به عنوان ورودی، شرح متناظر آن را در قالب یک جمله تولید و با گرفتن یک جمله به عنوان ورودی، تصویر مربوط به آن را با بازیابی نماید.
\\
در این روش با در نظر گرفتن واحد عصبی ارائه شده در پژوهش \cite{mikolov2010recurrent} و اضافه کردن دو متغیر دیگر به آن، مدل نهایی تولید شده است. متغیرهای اضافه شده به این مدل، شامل متغیری برای  بردار ویژگی تصویر و متغیر دیگر برای تفسیر بصری آخرین کلمه دیده شده، است.
\\
شبکه عصبی ارائه شده در این پژوهش، توزیع احتمال توام تصاویر و جملات را مدل‌سازی می‌نماید. در صورتی که جمله به عنوان ورودی داده شده باشد، توزیع احتمال تصویر به شرط جمله قابل محاسبه و تصویر مربوطه قابل بازیابی است. در صورتی‌که تصویر به عنوان ورودی داده شده باشد، توزیع احتمال جمله به شرط تصویر قابل محاسبه است.
\\
نتایج ارائه شده در این پژوهش، با روش‌های دیگر مقایسه شد. برای تولید جمله به شرط داشتن تصویر، میزان امتیاز \lr{BLEU} حاصل توسط مدل در بهترین حالت برای مجموعه‌داده \lr{Flickr8k} مقدار 13.1، برای مجموعه‌داده \lr{Flickr30k} مقدار 12.0 و برای مجموعه‌داده \lr{MS COCO} مقدار 18.8 بوده است. این در حالیست که نتایج حاصل برای مدل \lr{RNN + VGG} به ترتیب برابر با 12.4، 11.9 و 18.4 بوده و مقادیر به دست آمده برای جملاتی که توسط عوامل انسانی تولید شده‌اند به ترتیب برابر با 20.6، 18.9 و 19.2 بوده است. نتایج نشان می‌دهد، روش ارائه شده در حوزه تولید شرح متناظر تصاویر از روش‌های استاندارد دیگر بهتر بوده اما هنوز به جملات تولید شده توسط انسان نمی‌رسد.
\\
همین‌طور برای بازیابی تصاویر با داشتن جمله ورودی، نتایج حاصل توسط مدل برای مجموعه‌داده‌ \lr{Flickr30k} به ترتیب برای معیارهای \lr{R@1}، \lr{R@5}، \lr{R@10}  و \lr{Med r 500} در بهترین حالت برابر با 18.5، 45.7، 58.1 و 7 است. این در حالیست که نتایج حاصل توسط مدل \lr{RNN + VGG} به ترتیب برابر با 15.1، 41.1، 54.1 و 9 است.

\subsection{توجه بصری}

چارچوب کاری انکودر-دیکودر یکی از اصلی‌ترین چارچوب‌های کاری در حوزه ترجمه ماشینی و پیرو آن تولید شرح متناظر تصویر به شمار می‌رود. انکودر در این چارچوب کاری وظیفه نگاشت ورودی به فضای معنا و دیکودر وظیفه نگاشت فضای معنا به فضای خروجی را بر عهده دارد. در حوزه ترجمه ماشینی معمولا از یک شبکه عصبی حافظه کوتاه‌مدت بلند به عنوان دیکودر استفاده می‌شود. این شبکه عصبی با دریافت کلمات جمله ورودی به ترتیب، بردار حالت مخفی خود را به‌روز‌رسانی می‌نماید. در نهایت می‌توان از این بردار به عنوان بردار حاصل نگاشت جمله ورودی به فضای معنا استفاده نمود.
\\
دیکودر در این چارچوب کاری با دریافت بردار ویژگی تولید شده توسط دیکودر، عمل تولید خروجی را برعهده خواهد داشت. در حوزه ترجمه ماشینی معمولا یک شبکه عصبی بازگشتی برای دیکودر می‌تواند مورد استفاده قرار بگیرد. به طور معمول، بردار ویژگی تولید شده توسط انکودر، به عنوان یک ورودی به دیکودر داده می‌شود و دیکودر در هر مرحله با تولید یک کلمه به عنوان خروجی، بردار حالت مخفی خود را به‌روزرسانی نموده و با استفاده از بردار حالت مخفی جدید، اقدام به تولید کلمه جدید می‌نماید.
\\
یکی از محدودیت‌های جدی فرایند مذکور این است که بردار ویژگی فقط یک بردار با طول ثابت است و اولا انکودر باید بتواند تمام اطلاعات قابل استخراج را تنها در این بردار جاسازی نماید و ثانیا دیکودر باید بتواند تمام اطلاعات مورد نیاز خود برای تولید کلمه و جمله را فقط از همین یک بردار استخراج نماید. این مشکل، پژوهش‌گران را بر آن داشت تا بردار ویژگی را از یک بردار با طول ثابت به یک دنباله بردار با طول ثابت و تعداد متغیر تغییر دهند. 
\\
به بردارهای ویژگی تولید شده در حالت جدید، حاشیه‌نویسی می‌گویند. این حاشیه‌نویسی‌ها باید دارای دو شرط زیر باشند:
\begin{enumerate}
\item دربرگیرنده تمام معنای ورودی باشند.
\item تمرکز بیشتری روی معنای یک بخش مشخص از ورودی داشته باشند.
\end{enumerate}
با در نظر گرفتن این ویژگی‌ها، دیکودر قادر خواهد بود تا هنگام تولید هر کلمه، روی معنای یک بخش از جمله تمرکز بیشتری داشته باشد و فقط از آن بخش برای تولید کلمه استفاده نماید. به این شکل، کلمات تولید شده شباهت بیشتری به ورودی خواهند داشت و ترجمه‌های بهتری حاصل خواهد شد. 
\\
در سال 2015، آقای بنجیو و همکارانش در پژوهش \cite{xu2015show} روشی ارائه دادند که در آن برای اولین بار از ایده استفاده از نقطه توجه در حوزه ترجمه ماشینی برای تولید شرح متناظر تصویر استفاده نمودند. در این پژوهش، از یک شبکه عصبی کانولوشنی به عنوان انکودر استفاده شده است. خروجی شبکه از لایه ماقبل آخر گرفته شده که منجر به ایجاد تعداد زیادی بردار ویژگی از تصویر می‌شود که هر کدام از این بردارهای ویژگی، از یک ناحیه از تصویر ایجاد شده‌اند و تمرکز بیشتری روی آن ناحیه داشته‌اند. 
\\
بدین ترتیب با استفاده از یک شبکه عصبی بازگشتی به عنوان دیکودر و استفاده از بردارهای حاشیه‌نویسی ایجاد شده توسط انکودر می‌توان به راحتی عملیات تولید شرح متناظر تصویر را انجام داد. تنها نکته‌ای که باید مشخص شود، چگونگی استفاده از بردارهای حاشیه‌نویسی است. در این پژوهش دو روش مختلف برای استفاده از بردارهای حاشیه‌نویسی مطرح شده است.
\\
روش اول موسوم به روش توجه سخت، روشی است که در آن فقط یک بردار حاشیه‌نویسی انتخاب شده و از آن برای تولید جمله استفاده می‌شود. در این روش به هر یک از بردارهای حاشیه‌نویسی توسط یک مدل که قبلا آموزش دیده است، یک وزن اختصاص می‌دهیم و سپس با توجه به وزن‌های تخصیص داده شده به هر بردار حاشیه‌نویسی، یکی از آن‌ها را به عنوان بردار ویژگی تصویر انتخاب کرده و از آن در مراحل بعدی استفاده می‌کنیم.
\\
روش دوم موسوم به روش توجه نرم، روشی است که در آن یک بردار ویژگی کلی از روی بردارهای حاشیه‌نویسی تولید شده و از آن بردار در مراحل بعدی استفاده می‌شود. برای تولید این بردار نیز مانند روش توجه سخت، ابتدا توسط یک مدل که از پیش‌آموزش دیده است، به هر یک از بردارهای حاشیه‌نویسی یک وزن اختصاص می‌دهیم. سپس می‌توان با محاسبه امید ریاضی بردارهای حاشیه‌نویسی با توجه به وزن هر یک از آن‌ها بردار ویژگی نهایی را برای تصویر تولید و از آن برای تولید جمله استفاده کرد.
\\
آزمایشات انجام شده روی این مدل نشان می‌دهد، معیار \lr{BlEU-1} حاصل از این روش با استفاده از توجه سخت معمولا از مدل توجه نرم بیش‌تر بوده است. مطابق با نتایج گزارش شده در این پژوهش، میزان امتیاز \lr{BLEU-1} حاصل توسط توجه سخت روی مجموعه‌داده‌های \lr{MS COCO}، \lr{Flickr30k}  و \lr{Flickr8k} به ترتیب برابر با 71.8، 66.9 و 67.0 است. این در حالیست که امتیاز حاصل توسط توجه نرم روی همین مجموعه‌های داده، به ترتیب برابر با 70.8، 66.7 و 67.0 و امتیاز کسب شده توسط مدل \lr{Log Bilinear} در بهترین حالت، به ترتیب برابر با 70.8، 60.0 و 65.6 بوده است. 
\\
مطابق با آزمایشات انجام‌شده، استفاده از توجه نرم، معیار \lr{METEOR} را نسبت به استفاده از توجه سخت افزایش می‌دهد. طبق نتایج گزارش‌شده در پژوهش، امتیاز \lr{METEOR} حاصل از توجه نرم به ترتیب روی مجموعه‌داده‌های \lr{Flickr8k}، \lr{Flickr30k} و \lr{MS COCO} برابر با 18.93، 18.49 و 23.90 بوده است. این در حالیست که امتیاز کسب‌شده توسط توجه سخت به ترتیب برابر با 20.30، 18.46 و 23.04 و امتیاز کسب‌شده توسط روش \lr{Log Bilinear} در بهترین حالت به ترتیب برابر است با 17.31، 16.88 و 20.03. این موضوع نشان می‌دهد با وجود این‌که جملات تولید شده توسط روش توجه سخت با در نظر گرفتن جملات موجود در مجموعه‌داده از امتیاز بالاتری نسبت به جملات تولید شده توسط توجه نرم برخوردارند؛ استفاده از توجه نرم، منجر به تولید جملات قابل قبول‌تری توسط انسان می‌شود.
\\
پژوهش‌های مختلفی از این ایده در حوزه‌های مختلف استفاده نموده‌اند که گزارش مختصری از تعدادی از این پژوهش‌ها ارائه شده است.

\subsection{حافظه فعال}
در این بخش، یکی از جدید‌ترین روش‌هایی را که در حوزه ترجمه ماشینی مورد استفاده قرار می‌گیرد و در پژوهش \cite{lukas2017can} که در سال 2017 ارائه شده است، عمل‌کرد بهتری نسبت به روش‌های مبتنی بر نقطه توجه از خود نشان داده است را، که به نام حافظه فعال شناخته می‌شود، مورد بررسی قرار دادیم.  در ابتدا واحد بازگشتی‌ گیت‌دار و واحد بازگشتی گیت‌دار کانولوشنی معرفی شدند که روابط و ساختاری مشابه شبکه حافظه کوتاه‌مدت بلند دارند. سپس با استفاده از این واحد‌ها، اقدام به ساخت ساختار شبکه \lr{GPU} نمودیم.
\\
مطابق با پژوهش \cite{lukas2017can}، ساختار شبکه \lr{GPU} به همان شکل که ارائه شد، توان رقابت با مدل‌های مشابه دیگر را در حوزه تولید شرح متناظر تصویر، ندارد. به همین دلیل نسخه‌های مارکفی و توسعه‌یافته این شبکه را ارائه نمودیم که عمل‌کردهای بهتری از خود نشان دادند. این شبکه در حوزه یادگیری الگوریتم‌، به ويژه در حوزه یادگیری الگوریتم‌های ساده مانند ضرب و جمع اعداد بسیار بزرگ، عمل‌کرد بسیار خوبی  از خود نشان داده است.
\\
ایده حافظه فعال بر خلاف ایده روش‌های مبتنی بر توجه، بر این است که در هر مرحله از تولید خروجی، از تمام حافظه موجود استفاده نماییم. نکته‌ای که باعث کاهش چشم‌گیر عمل‌کرد این مدل در حوزه ترجمه ماشینی می‌شود، عدم وجود وابستگی کافی بین مولفه‌های خروجی شبکه است. با اعمال وابستگی‌های مارکفی، نسخه مارکفی این شبکه قابل حصول است که علاوه بر بهبود عمل‌کرد نسخه استاندارد، توانایی رقابت با مدل‌های مبتنی بر توجه را ندارد. در نسخه توسعه‌یافته این شبکه، وابستگی بین مولفه‌های خروجی، شدیدتر از وابستگی‌های مارکفی در نظر گرفته شده است که باعث افزایش چشم‌گیر کارایی مدل و رقابت‌پذیری مدل با نسخه‌های مبتنی بر توجه شده است. 
\\
مطابق با نتایج گزارش شده در پژوهش \cite{lukas2017can}، نسخه توسعه‌یافته شبکه قادر به دست‌یابی به امتیاز \lr{BLEU} برابر با 29.6 روی مجموعه‌داده جملات معادل زبان انگلیسی و فرانسوی شده است. این در حالیست که مدل مبتنی بر نقطه توجه امتیاز \lr{BLEU} برابر با 26.4 را کسب کرده است. این بهبود عمل‌کرد از در نظر گرفتن وابستگی بیشتر بین مولفه‌های خروجی حاصل شده است.
\\
مدل حافظه فعال می‌تواند با تعداد پارامترهای به مراتب کم‌تر نسبت به مدل‌های مبتنی بر نقطه توجه، عمل ترجمه ماشینی را انجام دهد. با این وجود هنوز مدل‌های مبتنی بر نقطه توجه که از توجه سخت استفاده می‌کنند عمل‌کرد‌های بهتری نسبت به مدل‌ حافظه فعال از خود نشان می‌دهند. 

\subsection{نتیجه‌گیری}

به دلیل پیچیدگی تحلیلی و عمل‌کردهای ضعیف، روش‌های مبتنی بر مدل‌های گرافی احتمالاتی، از سال 2013 به بعد کاربرد زیادی در بین پژوهش‌گران حوزه تولید شرح متناظر تصویر نداشتند و به طور کلی روش‌های مبتنی بر یادگیری عمیق، گوی سبقت را از روش‌های گرافی احتمالاتی ربوده‌اند. اما در میان روش‌های مبتنی بر یادگیری عمیق می‌توان پژوهش‌های موجود را به دو دسته کلی تقسیم‌بندی نمود.
\begin{enumerate}
\item  روش‌های استاندارد \\
این روش‌ها از ترکیب یک شبکه عصبی کانولوشنی با یک یا چند شبکه عصبی بازگشتی به منظور تولید شرح متناظر تصویر استفاده می‌نمایند. تعداد پژوهش‌هایی که در این دسته قرار می‌گیرند بسیار زیاد است و بخش قابل توجهی از این پژوهش‌ها با اعمال تغییرات کوچک در ساختار، تلاش برای دست‌یابی به دقت‌های بیشتر می‌نمایند. در بین پژوهش‌های موجود در این حوزه، می‌توان پژوهش \cite{karpathy2015deep}، که توسط خانم لی ارائه شده است، را به عنوان یکی از کامل‌ترین پژوهش‌ها معرفی نمود که در هر دو بخش درک صحنه و تولید جمله، نو‌آوری‌های زیادی داشته است. فرایند آموزش این پژوهش کمی پیچیده است. تعداد پارامترهای مدلی که در این پژوهش مطرح شده است، حدود 60 میلیون پارامتر است.
\\
\textbf{به نظر می‌رسد بتوان با ترکیب بخش‌هایی از این پژوهش‌ها با یک‌دیگر، روش جدیدی ارائه داد که به دقت‌های بهتری دست‌بیابد. به عنوان مثال، انتخاب چند کلیدواژه برای هر تصویر و مشروط کردن شبکه عصبی تولید‌کننده جمله، به تولید جملاتی که شامل این کلید‌واژه‌ها باشند، انتظار می‌رود بتواند بهبود خوبی در نتایج حاصل ایجاد نماید.}
\item  روش‌های مبتنی بر توجه بصری \\
این دسته‌ از پژوهش‌ها از سال 2015 با پژوهش آقای بنجیو \cite{xu2015show} به شکل جدی مطرح شدند. در این دسته از پژوهش‌ها، ارائه مکانیزم مناسب برای محاسبه نقطه توجه از اهمیت بالایی برخوردار است. تعداد پارامترهای مدلی که در این پژوهش ارائه شده است حدود 120 میلیون پارمتر است که منجر به نیاز به پردازنده گرافیکی قوی دارد. با این حال، فرایند آموزش این مدل‌ها ساده‌تر از مدل‌های قبلی است.
\end{enumerate}
\textbf{ 
یکی از ایده‌های دیگر برای نوآوری در پژوهش این است که از ایده حافظه فعال در زمینه تولید شرح متناظر تصویر استفاده شود. تا کنون پژوهشی از این ایده در حوزه تولید شرح متناظر تصویر استفاده نکرده است. با توجه به این‌که مقاله \cite{lukas2017can} در سال 2017 ارائه شده و مقایسه‌ای بین این روش و روش‌های مبتنی بر توجه ارائه داده است، احتمال این‌که پژوهش‌گر‌های دیگری اقدام به استفاده از آن در حوزه ترجمه ماشینی نمایند بسیار زیاد است. با این حال هنوز نکات مبهم بسیاری در این روش برای من وجود دارد. 
\\
علاوه بر این، ترکیب ایده حافظه فعال با روش‌های مبتنی بر توجه بصری با استفاده از یک ماسک روی حافظه می‌تواند مورد استفاده قرار بگیرد که در پژوهش \cite{lukas2017can} به عنوان روال‌های آتی مطرح شده است. ممکن است ارائه یک ترکیب مناسب و استفاده از آن در حوزه تولید شرح متناظر تصاویر بتواند نتایج قابل مشاهده‌ای را ارائه دهد.
}

